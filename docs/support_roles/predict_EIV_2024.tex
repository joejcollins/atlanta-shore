\documentclass{roles}

\title{Predicting biodiversity gains from beaver introduction}
\author{Joe Collins - j.collins@zengenti.com}
\date{11 December 2023}

\begin{document}

\maketitle

\section*{Background}

In 2019, the Spains Hall Estate, Essex,
initiated a nature recovery and natural flood management program by introducing beavers,
into a fenced enclosure 
\href{https://www.spainshallestate.co.uk/nfm\_beavers}{https://www.spainshallestate.co.uk/nfm\_beavers}.
Comprehensive monitoring of
the hydrology, canopy cover, and botanical species composition
has been conducted each year (2019-2023)
to assess the impact of beavers within the enclosure.
The data from these surveys is freely available at 
\href{https://github.com/joejcollins/atlanta-shore}{https://github.com/joejcollins/atlanta-shore}.
An additional survey will be conducted in mid summer 2024.

As part of the survey geo referenced images have been taken
of ground and canopy cover at each of the survey points 
\href{https://joejcollins.github.io/atlanta-shore/survey-points.html}{https://joejcollins.github.io/atlanta-shore/survey-points.html}.
The images do not provide sufficient detail to identify individual plants
but it is hoped that the images contain sufficient information to predict
the plant assemblages or Indicator Values 
\href{https://en.wikipedia.org/wiki/Indicator\_value}{https://en.wikipedia.org/wiki/Indicator\_value}
from features in the images.

The project uses R for statistical analysis and Python for data cleaning and preparation.
Some familiarity with Git is needed but training can be given.
The team includes the estate manager, two ecologists, a hydrologist and a software engineer,
and would welcome input from a data scientist.
\section*{Outcomes}

The project's objective is to develop a model
that can predict Indicator Values
\href{https://en.wikipedia.org/wiki/Indicator\_value}{https://en.wikipedia.org/wiki/Indicator\_value}
at the newly established beaver release sites.
These values serve as indicators of environmental conditions such as
soil moisture, nutrient levels, and light availability.
By leveraging machine learning techniques,
the model will be trained to forecast Indicator Values for arbitrary locations and time frames.

Once the model is trained and validated,
it will be deployed to generate annual predictions and
maps illustrating the anticipated changes in Indicator Values
at the release sites for the next five years.
These forecasts will provide valuable insights
into how the ecosystem is expected to evolve and adapt following the introduction of beavers.
Additionally, the project will compute confidence limits for these predictions,
providing users with an understanding of the uncertainty associated with the forecasts.

If possible, the project aims to package the model into a web-based tool or application.
This platform will allow users,
such as land managers to explore and interact with the predicted environmental changes.
Users should be able to visualise the forecasted Indicator Values spatially on maps,
examine temporal trends, and assess the potential impacts on ecosystem dynamics.

\section*{Team}

Whilst the Spains Hall estate is a commercial enterprise,
this is a volunteer lead study
run by a collaborative team consisting of:

\begin{description}
    \item[Sarah Brockless] Ecologist \href{https://www.linkedin.com/in/sarah-brockless-833291a7/}{https://www.linkedin.com/in/sarah-brockless-833291a7/}.
    \item[Joe Collins] Software Engineer \href{https://www.linkedin.com/in/joejcollins/}{https://www.linkedin.com/in/joejcollins/}.
    \item[Mags Cousins] Botanist \href{https://www.researchgate.net/profile/Mags-Cousins}{https://www.researchgate.net/profile/Mags-Cousins}
    \item[David Gasca] Hydrologist \href{https://www.linkedin.com/in/david-gasca-7830537/}{https://www.linkedin.com/in/david-gasca-7830537/}.
\end{description}

\end{document}
