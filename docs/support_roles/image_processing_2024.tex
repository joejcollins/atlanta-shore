\documentclass{roles}

\title{Image Analysis for Biodiversity Monitoring}
\author{Joe Collins}
\date{14 December 2023}

\begin{document}

\maketitle

\section*{Background}

In 2019, the Spains Hall Estate, Essex,
initiated a nature recovery and natural flood management program by introducing beavers,
into a fenced enclosure
\href{https://www.spainshallestate.co.uk/nfm\_beavers}{https://www.spainshallestate.co.uk/nfm\_beavers}.
Comprehensive monitoring of
the hydrology, canopy cover, and botanical species composition
has been conducted each year (2019-2023)
to assess the impact of beavers within the enclosure.
The data from these surveys is freely available at
\href{https://github.com/joejcollins/atlanta-shore}{https://github.com/joejcollins/atlanta-shore}.
An additional survey will be conducted in mid summer 2024.

As part of the survey geo referenced images have been taken
of ground and canopy cover at each of the survey points
\href{https://joejcollins.github.io/atlanta-shore/survey-points.html}{https://joejcollins.github.io/atlanta-shore/survey-points.html}.
The images do not provide sufficient detail to identify individual plants
but it is hoped that the images contain sufficient information to predict
the plant assemblages or Indicator Values
\href{https://en.wikipedia.org/wiki/Indicator\_value}{https://en.wikipedia.org/wiki/Indicator\_value}
from features in the images.

The project uses R for statistical analysis and Python for data cleaning and preparation.
Some familiarity with Git is needed but training can be given.
The team includes the estate manager, two ecologists, a hydrologist and a software engineer,
and would welcome input from a data scientist.

\section*{Outcomes}

The project aims to streamline botanical assessments by leveraging image data to predict Indicator Values
\href{https://en.wikipedia.org/wiki/Indicator\_value}{https://en.wikipedia.org/wiki/Indicator\_value} and plant assemblages.

The goal is to develop an algorithm that can accurately analyse images
to provide metrics for canopy and ground cover,
thus enabling less experienced individuals or
botanists with limited time to assess locations efficiently
based solely on gross/course image data.

Additionally, there is a plan to create a mobile application that offers basic metrics
for canopy and green ground cover.
Moreover,
this application will utilise image data to predict Indicator Values for specific locations.
Such a tool would serve as a convenient and expedited method
for verifying whether predictions align with actual observations on the ground.

\section*{Team}

Whilst the Spains Hall estate is a commercial enterprise,
this is a volunteer lead study
run by a collaborative team consisting of:

\begin{description}
    \item[Sarah Brockless] Ecologist \href{https://www.linkedin.com/in/sarah-brockless-833291a7/}{https://www.linkedin.com/in/sarah-brockless-833291a7/}.
    \item[Joe Collins] Software Engineer \href{https://www.linkedin.com/in/joejcollins/}{https://www.linkedin.com/in/joejcollins/}.
    \item[Mags Cousins] Botanist \href{https://www.researchgate.net/profile/Mags-Cousins}{https://www.researchgate.net/profile/Mags-Cousins}
    \item[David Gasca] Hydrologist \href{https://www.linkedin.com/in/david-gasca-7830537/}{https://www.linkedin.com/in/david-gasca-7830537/}.
\end{description}

\end{document}
