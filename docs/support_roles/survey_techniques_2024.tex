\documentclass{roles}

\title{Beavers and Botany Background}
\author{Joe Collins - j.collins@zengenti.com}
\date{14 December 2023}

\begin{document}

\maketitle

\section*{Beavers and Botany Background}

The Spains Hall Estate
(in Finchingfield, Essex)
is leading the field in demonstrating
how to create biodiversity gain on private land.
In 2019 the estate reintroduced beavers into a fenced enclosure
as part of their program of nature recovery and natural flood reduction.
The beaver enclosure is relatively confined
and the beavers have flourished,
so the site has provided an accelerated demonstration of the affects of the introduction of beavers
on biodiversity and water levels.
The effects of the beavers have been intensively studied,
using a wide variety of techniques including remote sensing, drone photography and invertebrate.
The program of study has included a detailed annual botanical survey
aimed at monitoring the effect of beaver introduction on the plant assemblages across the enclosure
and to evaluate botanical survey techniques for use at other locations in the UK.

In the summer of 2023 the Spains Hall Estate
introduced beavers in two much larger and more gently sloping sites enclosures.
Using data from the first beaver enclosure
it should be possible to predict the kinds of changes to be expected and the timeline for those changes.
In 2024 the estate will create Biodiversity Units that can be sold to developers 
who are unable to meet the 110\% biodiversity gain rules.
To do this confidently the estate needs to be able to 
provide accurately quantified predictions for the expected changes.

Whilst the Spains Hall estate is a commercial enterprise,
this is a volunteer lead study
run by a collaborative team of experts from the fields of hydrology, botany and software development.
These are:

Sarah Brockless - Ecologist <https://www.linkedin.com/in/sarah-brockless-833291a7/>.
Joe Collins - Software Engineer <https://www.linkedin.com/in/joejcollins/>.
Mags Cousins - Botanist <https://www.researchgate.net/profile/Mags-Cousins>.
Dave Gasca - Hydrologist <https://www.linkedin.com/in/david-gasca-7830537/>.

The team are seeking to develop and enhance these techniques
to help monitor and manage other aquatic sites in the UK.
Unfortunately, the team lacks expertise in statistics
and is looking for assistance to create statistical models for predicting the effects
on the two new beaver enclosures.

The annual botanical survey takes place in mid summer
so that the maximum amount of daylight is available
and the plants can be most easily identified.
The next survey is 20-24 June 2024,
and it is hoped that whoever joins the team
will take to opportunity to get involved in the practical.

The survey data and preliminary processing are available
to review at <https://github.com/joejcollins/atlanta-shore>.

\end{document}